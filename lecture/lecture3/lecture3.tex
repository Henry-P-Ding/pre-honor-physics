\documentclass[20pt]{beamer}

\usetheme{Berkeley}
\setbeamertemplate{caption}[numbered]

\usepackage{geometry}
\geometry{paperwidth=10in,paperheight=8in}
\usepackage[utf8]{inputenc}
\usepackage{amsmath}
\usepackage{amssymb}
\usepackage{amsthm}
\usepackage{physics}
\usepackage[version=4]{mhchem}
\usepackage{siunitx}
\usepackage{graphicx}
\graphicspath{{./figures/}}
\usepackage{hyperref}
\hypersetup{%
	colorlinks=true,
	urlcolor=cyan,
}

\usepackage{import}
\usepackage{xifthen}
\usepackage{pdfpages}
\usepackage{transparent}

\newcommand{\inkfig}[2]{%
	\def\svgwidth{#1}
	\import{./figures/}{#2.pdf_tex}
}

\newcommand{\mplfig}[1]{%
	\input{./figures/#1}
}

% matplotlib pgf output preamble
\usepackage{pgfplots}
\pgfplotsset{compat=1.18}
\def\mathdefault#1{#1}
\everymath=\expandafter{\the\everymath\displaystyle}

\ifdefined\pdftexversion\else  % non-pdftex case.
	\usepackage{fontspec}
\fi
\makeatletter\@ifpackageloaded{underscore}{}{\usepackage[strings]{underscore}}\makeatother

\newcommand{\diff}[1]{\operatorname{d}#1}
\renewcommand{\vec}[1]{\mathbf{#1}}


\author{Henry Ding}
\date{\today}
\title{Lesson 3: Newton's Laws}


\begin{document}

\frame{\titlepage}

\begin{frame}
	\frametitle{Homework Questions?}
\end{frame}

\section{Forces}

\begin{frame}
	\frametitle{Force}
	\begin{definition}
		A \textbf{force} is
	\end{definition}
\end{frame}

\begin{frame}
	\frametitle{Systems}
	Internal/External Forces
\end{frame}

\begin{frame}
	\frametitle{Types of Forces}
	Some examples of forces
	\begin{itemize}
		\item \textbf{Gravity}
		\item \textbf{Tension}
		\item \textbf{Friction}
	\end{itemize}
\end{frame}

\section{Newton's Laws}

\begin{frame}
	\frametitle{Newton's First Law}
	Newton's First Law
	Mass
\end{frame}

\begin{frame}
	\frametitle{Newton's Second Law}
	Newton's first law discuss objects moving at constant velocity, but they can also be accelerating...
	\begin{theorem}[Newton's Second Law]
		Test
		\begin{align*}
			\vec{F}_\mathrm{net} = \sum \vec{F} = m \vec{a}
		\end{align*}.
	\end{theorem}
\end{frame}

\begin{frame}
	\frametitle{Inertial Frames}
	Airplane example
\end{frame}

\begin{frame}
	\frametitle{Newton's Third Law}
\end{frame}

\section{Types of Forces}

\begin{frame}
	\frametitle{Gravitational Force}
	\begin{example}
		Free fall
	\end{example}
\end{frame}

\begin{frame}
	\frametitle{Normal Force}
	\begin{example}
		Book, earth, 3rd Law Pairs
	\end{example}
\end{frame}

\begin{frame}
	\frametitle{Friction}
	Static, Kinetic Friction
\end{frame}

\begin{frame}
	\frametitle{Tension}
	\begin{example}
		Holding up a block by rope
	\end{example}
\end{frame}

\begin{frame}
	\frametitle{Thrust}
	\begin{example}
		Rocket thrust
	\end{example}
\end{frame}

\section{Homework 3}

\begin{frame}
	\frametitle{Homework 3}
	\begin{block}{Textbook Problems}
		\begin{itemize}
			\item Test
		\end{itemize}
	\end{block}
\end{frame}


\end{document}
