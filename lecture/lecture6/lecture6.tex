\documentclass[20pt]{beamer}

\usetheme{Berkeley}
\setbeamertemplate{caption}[numbered]
\everymath=\expandafter{\the\everymath\displaystyle}

\ifdefined\pdftexversion\else  % non-pdftex case.
	\usepackage{fontspec}
\fi
\makeatletter\@ifpackageloaded{underscore}{}{\usepackage[strings]{underscore}}\makeatother

\newcommand{\diff}[1]{\operatorname{d}#1}
\renewcommand{\vec}[1]{\boldsymbol{#1}}


\author{Henry Ding}
\date{\today}
\title{Lesson 6: Momentum}


\begin{document}

\frame{\titlepage}

\section{Systems and Point Particles}

\begin{frame}
	\frametitle{Revisiting Newton's Laws}
	\begin{definition}
		A \textbf{point particle} is a point with some a position $\vec{x}$ and mass $m$. A point particle has no physical size or dimensions.
	\end{definition}
	\begin{alertblock}{Newton's Laws apply to point particles.}
		When we solve force problems using Newton's Laws, we are \textit{implicitly} modeling objects as point particles.
	\end{alertblock}
	\begin{definition}[Systems Revisited]
		I called a system a collection of objects we want to study. To be more precise, we can say a \textbf{system} is a collection of point particles.
	\end{definition}
	\begin{figure}[ht]
		\centering
		\inkfig{0.8\textwidth}{system}
		%\caption{}
		\label{fig:system}
	\end{figure}
\end{frame}

\begin{frame}
	\frametitle{System Example}
	Consider a rotating disk. We can break the disk up into very many point particles, each with their own position $\vec{x}$, velocity $\vec{v}$, and acceleration $\vec{a}$.
	Then, we can apply Newton's Laws to each particle in the disk.
	\begin{figure}[ht]
		\centering
		\inkfig{\textwidth}{disk}
		%\caption{}
		\label{fig:disk}
	\end{figure}
\end{frame}

\begin{frame}
	\frametitle{Everyday use of Momentum}
	\begin{itemize}
		\item ``The team is riding off of the momentum from our last win.``
		\item ``After securing a key endorsement, the political campaign starting gaining momentum.''
		\item ``The grassroots movement had too much momentum to stop now.''
	\end{itemize}
\end{frame}

\begin{frame}
	\frametitle{Physics Definition of Momentum}
	\begin{definition}
		For a point particle of mass $m$ and velocity $\vec{v}$, its \textbf{momentum} $\vec{p}$ is
		\begin{align*}
			\vec{p} = m \vec{v}.
		\end{align*}
	\end{definition}
	\begin{example}
		Recall the disk earlier. Find the total momentum of the disk system by adding together the momentum of every particle in the disk.
		\begin{figure}[ht]
			\centering
			\inkfig{0.7\textwidth}{disk}
			%\caption{}
			\label{fig:disk-example}
		\end{figure}
	\end{example}
\end{frame}

\begin{frame}
	\frametitle{Momentum Example}
	\begin{example}
		A $\SI{0.5}{\kilogram}$ block slides with velocity $\langle \SI{-6}{\meter/\second}, \SI{8}{\meter/\second} \rangle$.
		What is the block's momentum $\vec{p}$?
		What is the magnitude $p$?
	\end{example}
\end{frame}

\begin{frame}
	\frametitle{Change in Momentum}
	\begin{itemize}
		\item Sometimes, the mass of a system can change. For example, snow can pile onto the roof of a car, increasing its mass.
		\item If the mass of a system changes, the total momentum can change even if velocity of each particle does not. For example, the momentum of a car driving at a constant velocity can increase as snow piles onto the roof of a car.
	\end{itemize}
	Consider a one-dimensional particle's momentum at two points in time separated by a small time $\Delta t$ apart.
	\begin{itemize}
		\item The particle's mass barely changes by a small amount from $m$ to $m + \Delta m$
		\item The particle's velocity barely changes by a small amount from $v$ to $v + \Delta v$.
		\item The change in momentum is
		      \begin{align*}
			      \Delta p & = \left(m + \Delta m\right)\left(v + \Delta v\right) - mv \\
			               & = mv + v\Delta m + m \Delta v + \Delta m \Delta v - mv.
		      \end{align*}
		      $\Delta m \Delta v$ is an extremely small number, so we can ignore it! Then
		      \begin{align*}
			      \Delta p = v \Delta m + m \Delta v.
		      \end{align*}
	\end{itemize}
\end{frame}

\begin{frame}
	\frametitle{General Newton's Second Law}
	The small changes in momentum can be written as
	\begin{align*}
		\Delta p = m \Delta v + v \Delta m.
	\end{align*}
	The more general statement of Newton's Second Law is
	\begin{theorem}[Newton's Second Law]
		As the time $\Delta t$ becomes infinitely small (like instantaneous velocity or acceleration),
		\begin{align*}
			\vec{F}_\mathrm{net} = \frac{\Delta p}{\Delta t}
		\end{align*}
		or
		\begin{align*}
			\vec{F}_\mathrm{net} = \vec{v} \frac{\Delta m}{\Delta t} + m \vec{a}.
		\end{align*}
	\end{theorem}
	If the mass of our system does not change, then $\vec{F}_\mathrm{net} = m \vec{a}$ as usual! It turns out, we need extra force to change the mass of our system.
\end{frame}

\begin{frame}
	\frametitle{Collisions}
	\begin{example}
		Consider a block sliding towards another block on smooth ground (no friction) so that there are no \textit{external} horizontal forces.
		The blocks will collide and continue moving with some velocity.
		\begin{figure}[ht]
			\centering
			\inkfig{0.7\textwidth}{collision}
			%\caption{}
			\label{fig:collision}
		\end{figure}
		Let the collision force acting on blocks $1$ from $2$ be $\vec{F}_{2\rightarrow 1}$. By Newton's Third Law there is an equal and opposite force
		\begin{align*}
			\vec{F}_{1 \rightarrow 2} = - \vec{F}_{2 \rightarrow 1}.
		\end{align*}
	\end{example}
\end{frame}

\begin{frame}
	\frametitle{Conservation of Momentum}
	\begin{example}[cont.]
		Since, there are no external forces, the only two forces on our system are
		\begin{align*}
			\vec{F}_\mathrm{net} = \vec{F}_{1 \rightarrow 2} + \vec{F}_{2 \rightarrow 1} = 0.
		\end{align*}
		Then,
		\begin{align*}
			\vec{F}_\mathrm{net} = \frac{\Delta \vec{p}}{\Delta t} = 0
		\end{align*}
		so $\vec{p}$ is \textbf{conserved} (does not change with time).
	\end{example}
	We showed that $\vec{p}$ is conserved for two blocks colliding with no external forces, but this applies for any system as well.
	\begin{theorem}[Conservation of Momentum]
		Consider a system such that $\vec{F}_\mathrm{ext} = 0$. Then the system's total momentum $\vec{p}$ is conserved with time.
	\end{theorem}
\end{frame}

\begin{frame}
	\frametitle{Conservation of Momentum Example}
	\begin{example}
		A $\SI{3}{\kilogram}$ block slides to the right at $\SI{4}{\meter/\second}$ towards a stationary block of mass $\SI{2}{\kilogram}$.
		If the two blocks stick together after colliding, what are the final velocities of the two blocks?
	\end{example}
\end{frame}

\section{Homework 6}

\begin{frame}
	\frametitle{Homework 6}
	TODO
\end{frame}

\end{document}
