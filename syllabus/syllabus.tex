\documentclass[12pt]{article}

\usepackage{geometry}
\usepackage[utf8]{inputenc}
\usepackage{amsmath}
\usepackage{amssymb}
\usepackage{amsthm}
\usepackage{physics}
\usepackage{siunitx}
\usepackage{circuitikz}
\usepackage{graphicx}
\graphicspath{{./figures/}}
\usepackage{hyperref}
\hypersetup{%
	colorlinks=true,
	urlcolor=cyan,
}

\title{Pre-Honor Physics Syllabus}
\author{Henry Ding}
\date{\today}

\begin{document}
\maketitle
Pre-Honor Physics is designed for students in grades 8-11 who have completed Algebra 2 and possess a basic understanding of trigonometry and vectors, but have little to no prior experience in physics. This course builds a solid foundation in physics, develops an understanding of the fundamental laws behind natural phenomena, and inspires interdisciplinary interest. Pre-Honor Physics prepares students for success in Honor Physics. Topics include:
\begin{itemize}
	\item Math Review (trigonometry, vectors)
	\item Linear Motion
	\item Forces
	\item Rotational Motion
	\item Rotational Dynamics
\end{itemize}

\section*{Logistics}
\begin{itemize}
	\item \textbf{Lectures}: Tuesday, Thursday, Saturday; 1 pm to 3 pm Pacific Time on Zoom (\href{https://us06web.zoom.us/j/84225321342}{link})
	\item \textbf{Instructor}: Henry Ding (he/him) henry.d@princeton.edu. Feel free to contact me via email at any time. I will do my best to respond in a timely manner.
	\item \textbf{Textbook}s: \href{https://openstax.org/details/books/physics}{OpenStax Physics (High School)}, \href{https://openstax.org/details/books/precalculus-2e}{OpenStax Precalculus 2e}. Recommended readings provided. Only read as necessary, and do not read everything!
	\item \textbf{Assignments}: Homework is due at 11:59 PM via email the day before the next lesson. Assignment feedback will be available before the next lesson.
	\item \textbf{Language}: All lectures and class materials are in English.
\end{itemize}

\section*{Course Schedule}
There are 8 total lessons covering
\begin{enumerate}
	\item Math Review and Measurement
	\item Motion in One Dimension
	\item Newton's laws
	\item Motion in Two/Three Dimensions
	\item Applications of Newton's Laws
	\item Momentum
	\item Rotational Motion
	\item Rotational Dynamics
\end{enumerate}


\end{document}
